% GitHub Repo and Documentation: https://github.com/celiobjunior/resume-template
% Copyright © 2025 Celio B Junior. All rights reserved.
% 
% Licensed under the Apache License, Version 2.0 (the "License");
% you may not use this file except in compliance with the License.
% You may obtain a copy of the License at
%
%     http://www.apache.org/licenses/LICENSE-2.0
%
% This template follows best practices from README.md

% Início do documento LaTeX: define tipo e formato do documento
% a4paper = tamanho A4, 10pt = tamanho base da fonte
\documentclass[a4paper,10pt]{article}

% --- PACOTES ---
\usepackage[utf8]{inputenc}
\usepackage[T1]{fontenc}
\usepackage[portuguese]{babel}
\usepackage{geometry}
\usepackage{parskip}
\usepackage{hyperref}
\usepackage{titlesec}
% Para quebras de linha em URLs longas (não use \href{}, use \url{} para URLs longas)
% \usepackage{xurl}

% --- CONFIGURAÇÃO DO DOCUMENTO ---
% Define margens da página para maximizar espaço de conteúdo
\geometry{top=1.0cm, bottom=1.0cm, left=1.0cm, right=1.0cm}

% Remove números de página e cabeçalhos para visual limpo do currículo
\pagestyle{empty}

% Metadados do PDF - personalize com suas informações
\hypersetup{
    pdftitle={CV Seu Nome},
    pdfauthor={Seu Nome},
    colorlinks=true,
    linkcolor=black,
    urlcolor=black,
    citecolor=black,
    bookmarksdepth=1 
}

% Desabilita numeração das sections
\setcounter{secnumdepth}{0}

% Formata os cabeçalhos de cada section e coloca uma linha em baixo
\titleformat{\section}
{\Large\bfseries}
{}
{0em}
{}
[\titlerule\vspace{0.5ex}]

% --- INÍCIO DO DOCUMENTO ---
\begin{document}

% --- CABEÇALHO ---
% Substitua com suas informações pessoais
\begin{center}
    {\LARGE \textbf{Nome Completo}} 
    \\ [0.1cm]
    Cidade, Estado
    {\textbullet}
    Email: \href{mailto:seu.email@exemplo.com}{seu.email@exemplo.com} 
    {\textbullet}
    \href{https://www.linkedin.com/in/usuario}{linkedin.com/in/usuario} 
    {\textbullet}
    \href{https://github.com/usuario}{github.com/usuario}
\end{center}

% --- SEÇÕES ---

\section{Atividades de Liderança} 
    % Inclua trabalho voluntário, ligas acadêmicas ou cargos de liderança
    \subsection*{\texorpdfstring{
            \textbf{Nome da Empresa/Organização} \hfill Localização (ex: Remoto ou Estado, País)
        }{
            Nome da Empresa (Liderança) -- Localização
        }}
    \textit{Seu Cargo/Posição \hfill Data de Início - Data de Término (ex: Jan 2024 - Atual)}
        \begin{itemize}
            % Use verbos de ação do README - quantifique seu impacto quando possível
            \item Descreva uma conquista significativa de liderança com métricas específicas.
            \item Destaque como você gerenciou equipes, liderou iniciativas ou impulsionou resultados.
            \item Mostre impacto na organização ou comunidade que você serviu.
        \end{itemize}

% ------

\section{Experiência}
    % Liste as suas experiência de trabalho ou acadêmicas, mais recente primeiro
    \subsection*{\texorpdfstring{
            \textbf{Nome da Empresa/Organização} \hfill Localização (ex: Remoto ou Estado, País)
        }{
            Nome da Empresa -- Localização
        }}
    \textit{Cargo (ex: Engenheiro de Software, Analista de Dados) \hfill Data de Início - Data de Término}
        \begin{itemize} 
            % Foque em conquistas, não apenas na parte técnica - use o método STAR (Situação, Tarefa, Ação, Resultado)
            \item Alcancei [resultado específico] através de [ação tomada], resultando em [impacto quantificado].
            \item Desenvolvi/Construí/Criei [o que você fez] usando [tecnologias/métodos], resultando em [benefício].
            \item Colaborei com [tamanho/tipo da equipe] para [alcançar o quê], melhorando [métrica] em [porcentagem/quantidade].
        \end{itemize}

    \subsection*{\texorpdfstring{
        	\textbf{Nome da Empresa/Organização} \hfill Localização (ex: Remoto ou Estado, País)
        }{
            Nome da Empresa -- Localização
        }}
    \textit{Cargo (ex: Desenvolvedor Júnior, Assistente de Pesquisa) \hfill Data de Início - Data de Término}
        \begin{itemize}
            \item Use verbos ativos da tabela de verbos de ação no README.md
            \item Quantifique suas conquistas sempre que possível (números, porcentagens, prazos)
            \item Mostre progressão e responsabilidade crescente entre os cargos
        \end{itemize}
        
    \subsection*{\texorpdfstring{
        	\textbf{Universidade/Instituição de Pesquisa} \hfill Localização (ex: Remoto ou Estado, País)
        }{
            Universidade/Instituição de Pesquisa -- Localização
        }}
    \textit{Função (ex: Pesquisador de Graduação, Monitor) \hfill Data de Início - Data de Término}
        \begin{itemize}
            \item Inclua pesquisa, trabalho acadêmico ou projetos que demonstrem habilidades técnicas
            \item Destaque publicações, apresentações ou descobertas significativas
        \end{itemize}

% ------

\section{Habilidades}
    % Mantenha esta seção concisa e use o máximo de palavras chaves
    % que fazem sentido para a vaga que você deseja.
    \begin{itemize}
        \item \textbf{Técnicas:} Liste suas linguagens de programação, frameworks, ferramentas e metodologias
        \item \textbf{Idiomas:} Português (Nativo), Inglês (Nível de Proficiência), Terceiro Idioma (Nível de Proficiência)
    \end{itemize}

% ------

% !!!!!!!!!!!!!!!!!!!!!!
% Mude esta section para o começo, depois do header e antes
% das suas experiências se você está procurando sua primeira
% vaga ou algum estágio. Do contrário, deixe como está.
% !!!!!!!!!!!!!!!!!!!!!!

\section{Educação}
    % Formação mais recente primeiro
    \subsection*{\texorpdfstring{
            \textbf{Nome da Sua Universidade} \hfill Localização (ex: Remoto ou Estado, País)
        }{
            Nome da Sua Universidade (Educação) -- Localização
        }}
    \textit{Título do Seu Curso (ex: Bacharelado em Ciência da Computação) \hfill Data de Início - Data de Término (ex: Jan 2020 - Dez 2024)}

\end{document}
